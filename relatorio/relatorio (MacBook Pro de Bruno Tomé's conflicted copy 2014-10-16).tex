\documentclass[12pt]{article}
\usepackage{sbc-template}
\usepackage{graphicx,url}
\usepackage[brazil]{babel}   
\usepackage[utf8]{inputenc} 
\usepackage{amsmath}
\usepackage{mathtools}
\setlength\parindent{0pt}
\sloppy
\title{Engenharia de Software I - Extreme Programming (XP)}
\author{Bruno Tomé\inst{1}, Vinícius Laet\inst{1}, Ronan Nunes\inst{1}}

\address{Instituto Federal de Minas Gerais
  (IFMG)\\
  São Luiz Gonzaga, s/nº - Formiga / MG  - Brasil
  \email{ibrunotome@gmail.com, viniciusbatista023@gmail.com, ronannc1@yahoo.com}
}

\begin{document} 

\maketitle

\begin{abstract}
Report about the agile method eXtreme Programming (XP)
\end{abstract}
     
\begin{resumo} 
Relatório sobre o método ágil Programação Extrema (XP)
\end{resumo}

\section{Definição do processo}

Programação eXtrema (XP) é uma técnica revolucionária de desenvolvimento de software que se opõe a uma série de premissas adotadas pelos métodos tradicionais de engenharia de software. XP consiste numa série de práticas e regras que permitem aos programadores desenvolver software de alta qualidade de uma forma dinâmica e muito ágil.

\section{Características do processo}

\begin{listing}
\item{Design Simples}
\item{Design Simples}
\item{Design Simples}
\item{Design Simples}
\item{Design Simples}
\item{Design Simples}
\item{Design Simples}
\item{Design Simples}
\item{Design Simples}
\end{listing}
\end{document}
